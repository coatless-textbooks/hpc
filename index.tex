% Options for packages loaded elsewhere
\PassOptionsToPackage{unicode}{hyperref}
\PassOptionsToPackage{hyphens}{url}
\PassOptionsToPackage{dvipsnames,svgnames,x11names}{xcolor}
%
\documentclass[
  letterpaper,
  DIV=11,
  numbers=noendperiod]{scrreport}

\usepackage{amsmath,amssymb}
\usepackage{lmodern}
\usepackage{iftex}
\ifPDFTeX
  \usepackage[T1]{fontenc}
  \usepackage[utf8]{inputenc}
  \usepackage{textcomp} % provide euro and other symbols
\else % if luatex or xetex
  \usepackage{unicode-math}
  \defaultfontfeatures{Scale=MatchLowercase}
  \defaultfontfeatures[\rmfamily]{Ligatures=TeX,Scale=1}
\fi
% Use upquote if available, for straight quotes in verbatim environments
\IfFileExists{upquote.sty}{\usepackage{upquote}}{}
\IfFileExists{microtype.sty}{% use microtype if available
  \usepackage[]{microtype}
  \UseMicrotypeSet[protrusion]{basicmath} % disable protrusion for tt fonts
}{}
\makeatletter
\@ifundefined{KOMAClassName}{% if non-KOMA class
  \IfFileExists{parskip.sty}{%
    \usepackage{parskip}
  }{% else
    \setlength{\parindent}{0pt}
    \setlength{\parskip}{6pt plus 2pt minus 1pt}}
}{% if KOMA class
  \KOMAoptions{parskip=half}}
\makeatother
\usepackage{xcolor}
\setlength{\emergencystretch}{3em} % prevent overfull lines
\setcounter{secnumdepth}{5}
% Make \paragraph and \subparagraph free-standing
\ifx\paragraph\undefined\else
  \let\oldparagraph\paragraph
  \renewcommand{\paragraph}[1]{\oldparagraph{#1}\mbox{}}
\fi
\ifx\subparagraph\undefined\else
  \let\oldsubparagraph\subparagraph
  \renewcommand{\subparagraph}[1]{\oldsubparagraph{#1}\mbox{}}
\fi

\usepackage{color}
\usepackage{fancyvrb}
\newcommand{\VerbBar}{|}
\newcommand{\VERB}{\Verb[commandchars=\\\{\}]}
\DefineVerbatimEnvironment{Highlighting}{Verbatim}{commandchars=\\\{\}}
% Add ',fontsize=\small' for more characters per line
\usepackage{framed}
\definecolor{shadecolor}{RGB}{241,243,245}
\newenvironment{Shaded}{\begin{snugshade}}{\end{snugshade}}
\newcommand{\AlertTok}[1]{\textcolor[rgb]{0.68,0.00,0.00}{#1}}
\newcommand{\AnnotationTok}[1]{\textcolor[rgb]{0.37,0.37,0.37}{#1}}
\newcommand{\AttributeTok}[1]{\textcolor[rgb]{0.40,0.45,0.13}{#1}}
\newcommand{\BaseNTok}[1]{\textcolor[rgb]{0.68,0.00,0.00}{#1}}
\newcommand{\BuiltInTok}[1]{\textcolor[rgb]{0.00,0.23,0.31}{#1}}
\newcommand{\CharTok}[1]{\textcolor[rgb]{0.13,0.47,0.30}{#1}}
\newcommand{\CommentTok}[1]{\textcolor[rgb]{0.37,0.37,0.37}{#1}}
\newcommand{\CommentVarTok}[1]{\textcolor[rgb]{0.37,0.37,0.37}{\textit{#1}}}
\newcommand{\ConstantTok}[1]{\textcolor[rgb]{0.56,0.35,0.01}{#1}}
\newcommand{\ControlFlowTok}[1]{\textcolor[rgb]{0.00,0.23,0.31}{#1}}
\newcommand{\DataTypeTok}[1]{\textcolor[rgb]{0.68,0.00,0.00}{#1}}
\newcommand{\DecValTok}[1]{\textcolor[rgb]{0.68,0.00,0.00}{#1}}
\newcommand{\DocumentationTok}[1]{\textcolor[rgb]{0.37,0.37,0.37}{\textit{#1}}}
\newcommand{\ErrorTok}[1]{\textcolor[rgb]{0.68,0.00,0.00}{#1}}
\newcommand{\ExtensionTok}[1]{\textcolor[rgb]{0.00,0.23,0.31}{#1}}
\newcommand{\FloatTok}[1]{\textcolor[rgb]{0.68,0.00,0.00}{#1}}
\newcommand{\FunctionTok}[1]{\textcolor[rgb]{0.28,0.35,0.67}{#1}}
\newcommand{\ImportTok}[1]{\textcolor[rgb]{0.00,0.46,0.62}{#1}}
\newcommand{\InformationTok}[1]{\textcolor[rgb]{0.37,0.37,0.37}{#1}}
\newcommand{\KeywordTok}[1]{\textcolor[rgb]{0.00,0.23,0.31}{#1}}
\newcommand{\NormalTok}[1]{\textcolor[rgb]{0.00,0.23,0.31}{#1}}
\newcommand{\OperatorTok}[1]{\textcolor[rgb]{0.37,0.37,0.37}{#1}}
\newcommand{\OtherTok}[1]{\textcolor[rgb]{0.00,0.23,0.31}{#1}}
\newcommand{\PreprocessorTok}[1]{\textcolor[rgb]{0.68,0.00,0.00}{#1}}
\newcommand{\RegionMarkerTok}[1]{\textcolor[rgb]{0.00,0.23,0.31}{#1}}
\newcommand{\SpecialCharTok}[1]{\textcolor[rgb]{0.37,0.37,0.37}{#1}}
\newcommand{\SpecialStringTok}[1]{\textcolor[rgb]{0.13,0.47,0.30}{#1}}
\newcommand{\StringTok}[1]{\textcolor[rgb]{0.13,0.47,0.30}{#1}}
\newcommand{\VariableTok}[1]{\textcolor[rgb]{0.07,0.07,0.07}{#1}}
\newcommand{\VerbatimStringTok}[1]{\textcolor[rgb]{0.13,0.47,0.30}{#1}}
\newcommand{\WarningTok}[1]{\textcolor[rgb]{0.37,0.37,0.37}{\textit{#1}}}

\providecommand{\tightlist}{%
  \setlength{\itemsep}{0pt}\setlength{\parskip}{0pt}}\usepackage{longtable,booktabs,array}
\usepackage{calc} % for calculating minipage widths
% Correct order of tables after \paragraph or \subparagraph
\usepackage{etoolbox}
\makeatletter
\patchcmd\longtable{\par}{\if@noskipsec\mbox{}\fi\par}{}{}
\makeatother
% Allow footnotes in longtable head/foot
\IfFileExists{footnotehyper.sty}{\usepackage{footnotehyper}}{\usepackage{footnote}}
\makesavenoteenv{longtable}
\usepackage{graphicx}
\makeatletter
\def\maxwidth{\ifdim\Gin@nat@width>\linewidth\linewidth\else\Gin@nat@width\fi}
\def\maxheight{\ifdim\Gin@nat@height>\textheight\textheight\else\Gin@nat@height\fi}
\makeatother
% Scale images if necessary, so that they will not overflow the page
% margins by default, and it is still possible to overwrite the defaults
% using explicit options in \includegraphics[width, height, ...]{}
\setkeys{Gin}{width=\maxwidth,height=\maxheight,keepaspectratio}
% Set default figure placement to htbp
\makeatletter
\def\fps@figure{htbp}
\makeatother

\KOMAoption{captions}{tableheading}
\makeatletter
\makeatother
\makeatletter
\@ifpackageloaded{bookmark}{}{\usepackage{bookmark}}
\makeatother
\makeatletter
\@ifpackageloaded{caption}{}{\usepackage{caption}}
\AtBeginDocument{%
\ifdefined\contentsname
  \renewcommand*\contentsname{Table of contents}
\else
  \newcommand\contentsname{Table of contents}
\fi
\ifdefined\listfigurename
  \renewcommand*\listfigurename{List of Figures}
\else
  \newcommand\listfigurename{List of Figures}
\fi
\ifdefined\listtablename
  \renewcommand*\listtablename{List of Tables}
\else
  \newcommand\listtablename{List of Tables}
\fi
\ifdefined\figurename
  \renewcommand*\figurename{Figure}
\else
  \newcommand\figurename{Figure}
\fi
\ifdefined\tablename
  \renewcommand*\tablename{Table}
\else
  \newcommand\tablename{Table}
\fi
}
\@ifpackageloaded{float}{}{\usepackage{float}}
\floatstyle{ruled}
\@ifundefined{c@chapter}{\newfloat{codelisting}{h}{lop}}{\newfloat{codelisting}{h}{lop}[chapter]}
\floatname{codelisting}{Listing}
\newcommand*\listoflistings{\listof{codelisting}{List of Listings}}
\makeatother
\makeatletter
\@ifpackageloaded{caption}{}{\usepackage{caption}}
\@ifpackageloaded{subcaption}{}{\usepackage{subcaption}}
\makeatother
\makeatletter
\@ifpackageloaded{tcolorbox}{}{\usepackage[many]{tcolorbox}}
\makeatother
\makeatletter
\@ifundefined{shadecolor}{\definecolor{shadecolor}{rgb}{.97, .97, .97}}
\makeatother
\makeatletter
\makeatother
\ifLuaTeX
  \usepackage{selnolig}  % disable illegal ligatures
\fi
\IfFileExists{bookmark.sty}{\usepackage{bookmark}}{\usepackage{hyperref}}
\IfFileExists{xurl.sty}{\usepackage{xurl}}{} % add URL line breaks if available
\urlstyle{same} % disable monospaced font for URLs
\hypersetup{
  pdftitle={High Performance Computing},
  pdfauthor={James Joseph Balamuta},
  colorlinks=true,
  linkcolor={blue},
  filecolor={Maroon},
  citecolor={Blue},
  urlcolor={Blue},
  pdfcreator={LaTeX via pandoc}}

\title{High Performance Computing}
\author{James Joseph Balamuta}
\date{1/6/23}

\begin{document}
\maketitle
\ifdefined\Shaded\renewenvironment{Shaded}{\begin{tcolorbox}[borderline west={3pt}{0pt}{shadecolor}, frame hidden, interior hidden, sharp corners, breakable, boxrule=0pt, enhanced]}{\end{tcolorbox}}\fi

\renewcommand*\contentsname{Table of contents}
{
\hypersetup{linkcolor=}
\setcounter{tocdepth}{2}
\tableofcontents
}
\bookmarksetup{startatroot}

\hypertarget{welcome-to-hpc}{%
\chapter{Welcome to HPC}\label{welcome-to-hpc}}

The goal of High Performance Computing on a Cluster with \emph{R} is to
inform and distribute various recipes and techniques for performing
computational-intensive work on remote computing resources.

\hypertarget{thanks}{%
\subsection{Thanks!}\label{thanks}}

This work made use of the Illinois Campus Cluster, a computing resource
that is operated by the
\href{https://campuscluster.illinois.edu/}{Illinois Campus Cluster
Program (ICCP)} in conjunction with the
\href{https://ncsa.illinois.edu}{National Center for Supercomputing
Applications (NCSA)} which is supported by funds from the
\href{https://illinois.edu}{University of Illinois at Urbana-Champaign}.

Conversations with the ICCP staff has also greatly helped in developing
material. In particular, I would like to thank:

\begin{itemize}
\tightlist
\item
  Weddie Jackson
\item
  Matthew Long
\item
  Chit Khin
\end{itemize}

\part{Overview of Cluster Computing}

\hypertarget{cluster-computing}{%
\chapter{Cluster Computing}\label{cluster-computing}}

\hypertarget{what-is-cluster-computing}{%
\section{What is Cluster Computing?}\label{what-is-cluster-computing}}

Definition: Cluster

A \textbf{cluster} is a \emph{set of computers} that are connected
together and share resources as if they were one gigantic computer.

\hypertarget{how-does-cluster-computing-work}{%
\section{How Does Cluster Computing
WorK?}\label{how-does-cluster-computing-work}}

Definition: Parallel Processing

\textbf{Parallel Processing} is the act of carrying out multiple tasks
simultaneously to solve a problem. This is accomplished by dividing the
problem into independent subparts, which are then solved concurrently.

Definition: Jobs

\textbf{Jobs} denote the independent subparts.

\hypertarget{why-use-cluster-computing}{%
\section{Why use Cluster Computing?}\label{why-use-cluster-computing}}

Pros

\begin{itemize}
\tightlist
\item
  Speeds up simulations by allowing iterations to be run simultaneously.
\item
  Provides more resources for computations.

  \begin{itemize}
  \tightlist
  \item
    e.g.~CPU Cores, RAM, Hard Drive Space, and Graphics Cards (GPUs).
  \end{itemize}
\item
  Nightly snapshots/backups of files.
\item
  Extends the lifespan of your computer.
\end{itemize}

Cons

\begin{itemize}
\tightlist
\item
  Simulations are \textbf{not} instantly run.

  \begin{itemize}
  \tightlist
  \item
    Need to ``queue'' for resources.
  \end{itemize}
\item
  Higher barrier of entry due to knowledge requirements.
\item
  Poorly handles opening and closing data sets.
\item
  Adding or updating software is complex.
\end{itemize}

\hypertarget{cluster-software}{%
\chapter{Cluster Software}\label{cluster-software}}

\hypertarget{software-modules}{%
\section{Software Modules}\label{software-modules}}

Unlike a traditional desktop, you must load the different software that
you wish to use into the environment via \texttt{modulefiles}. The list
of supported software can be found on
\href{https://campuscluster.illinois.edu/resources/software/}{Software
List} or by typing:

\begin{Shaded}
\begin{Highlighting}[]
\ExtensionTok{module}\NormalTok{ avail}
\end{Highlighting}
\end{Shaded}

\hypertarget{viewing-retrieving-and-disabling-module-software}{%
\section{Viewing, Retrieving, and Disabling Module
Software}\label{viewing-retrieving-and-disabling-module-software}}

The most frequently used module commands are:

\begin{Shaded}
\begin{Highlighting}[]
\ExtensionTok{module}\NormalTok{ list              }\CommentTok{\# See active software modules}
\ExtensionTok{module}\NormalTok{ load }\OperatorTok{\textless{}}\NormalTok{software}\OperatorTok{\textgreater{}}\NormalTok{   \# Enable software}
\ExtensionTok{module}\NormalTok{ unload }\OperatorTok{\textless{}}\NormalTok{software}\OperatorTok{\textgreater{}}\NormalTok{ \# Disable software}
\ExtensionTok{module}\NormalTok{ purge             }\CommentTok{\# Removes all active modules}
\end{Highlighting}
\end{Shaded}

Replace \texttt{\textless{}software\textgreater{}} with the name of the
desired software module from \texttt{module\ avail}.

\hypertarget{latest-version-of-r}{%
\section{\texorpdfstring{Latest Version of
\emph{R}}{Latest Version of R}}\label{latest-version-of-r}}

As of \textbf{September 2021}, the latest version of \emph{R} on ICC is
\emph{R} \textbf{4.1.1}. We recommend using the latest version of R with
the \texttt{\_sandybridge} suffix. The reason for using
\texttt{\_sandybridge} is to ensure compatibility on older nodes inside
of the \texttt{stat} partition. For an example of a compatibility error,
please see (\textbf{debugging-errors?}).

Moreover, with this version, the default library does not contain any
non-standard packages.

\emph{R} can be accessed by using:

\begin{Shaded}
\begin{Highlighting}[]
\CommentTok{\# Load software}
\ExtensionTok{module}\NormalTok{ load R/4.1.1\_sandybridge}
\end{Highlighting}
\end{Shaded}

\textbf{Note:} If the version is not specified during the load,
e.g.~\texttt{module\ load\ R}, then a default version of \emph{R} will
be used. This default may change without warning.

Once \emph{R} is loaded, the Terminal/non-GUI version of \emph{R} can be
started by typing:

\begin{Shaded}
\begin{Highlighting}[]
\ExtensionTok{R}
\end{Highlighting}
\end{Shaded}

To exit an \emph{R} session on the cluster, type inside \emph{R}:

\begin{verbatim}
q(save = "no")
\end{verbatim}

This will terminate the \emph{R} session without saving any environment
values.

\hypertarget{ask-for-help}{%
\section{Ask for Help}\label{ask-for-help}}

ICC's help desk (via
\href{mailto:help@campuscluster.illinois.edu}{\nolinkurl{help@campuscluster.illinois.edu}})
can help install software on ICC. Please send them an e-mail and
\emph{CC} your advisor.

\hypertarget{writing-a-custom-module}{%
\subsection{Writing a Custom Module}\label{writing-a-custom-module}}

It is possible to compile and create your own modules. For details, see
the tutorial
\href{http://thecoatlessprofessor.com/programming/a-modulefile-approach-to-compiling-r-on-a-cluster/}{A
Modulefile Approach to Compiling \emph{R} on a Cluster}.

\hypertarget{cluster-storage}{%
\chapter{Storage}\label{cluster-storage}}

For additional details related to the illinois campus cluster, please
the
\href{https://campuscluster.illinois.edu/resources/docs/storage-and-data-guide/}{Storage
and Data Guide}

\hypertarget{storing-data-code}{%
\section{Storing Data \& Code}\label{storing-data-code}}

\begin{itemize}
\tightlist
\item
  Home Directory \texttt{\textasciitilde{}/}

  \begin{itemize}
  \tightlist
  \item
    Up to \textbf{\textasciitilde5GB} (Soft cap) /
    \textbf{\textasciitilde7GB} (Hard cap) with \textbf{nightly
    backups}.
  \item
    Storage is \textbf{private}.
  \end{itemize}
\item
  Project Spaces \texttt{/projects/stat/shared/\$USER}

  \begin{itemize}
  \tightlist
  \item
    \textbf{\textasciitilde21TB} of shared space with \textbf{nightly
    backups}.
  \item
    Storage is \textbf{shared} among \texttt{stat} members.
  \end{itemize}
\item
  Temporary Networked Storage \texttt{/scratch}

  \begin{itemize}
  \tightlist
  \item
    \textbf{\textasciitilde10TB} of space purged after \textbf{30 days}
    with \textbf{no backup}.
  \item
    Storage is \textbf{private}.
  \end{itemize}
\end{itemize}

\textbf{Soft caps}: gently warn the user to lower their storage size.
\textbf{Hard caps}: prevent the user from adding new files.

\hypertarget{backups}{%
\section{Backups}\label{backups}}

\hypertarget{backup-info}{%
\subsection{Backup Info}\label{backup-info}}

\begin{itemize}
\tightlist
\item
  \textbf{Daily} night time backups.
\item
  \textbf{30 days} of backups exist.
\item
  \textbf{No off-site backups for disaster recovery.}
\end{itemize}

\hypertarget{location-of-backups}{%
\subsection{Location of Backups}\label{location-of-backups}}

\begin{itemize}
\tightlist
\item
  Home Directory \texttt{\textasciitilde{}/}
\end{itemize}

\begin{Shaded}
\begin{Highlighting}[]
\ExtensionTok{/gpfs/iccp/home/.snapshots/home\_YYYYMMDD*/}\VariableTok{$USER}
\end{Highlighting}
\end{Shaded}

\begin{itemize}
\tightlist
\item
  Project Directory \texttt{/projects/stat/shared/\$USER}
\end{itemize}

\begin{Shaded}
\begin{Highlighting}[]
\ExtensionTok{/gpfs/iccp/projects/stat/.snapshots/statistics\_YYYYMMDD*}
\end{Highlighting}
\end{Shaded}

\hypertarget{cluster-setup}{%
\chapter{Cluster Setup}\label{cluster-setup}}

Within this chapter, we will cover establishing a workspace on the
Campus Cluster. Workspace setup usually requires about 5 different
steps.

\begin{itemize}
\tightlist
\item
  Ensure the cluster can easily be accessed from a local computer.
\item
  Enable command shortcuts through aliases.
\item
  Setup a GitHub access token for pulling software in from private
  repositories (skip if not needed).
\item
  Create a space on a project drive for where R packages should be
  installed.
\item
  Install \emph{R} packages!
\end{itemize}

\hypertarget{secure-shell-ssh-setup}{%
\section{Secure Shell (SSH) Setup}\label{secure-shell-ssh-setup}}

For accessing a cluster from command line, \textbf{Secure Shell (SSH)}
is preferred. Access to the cluster requires typing out each time:

\begin{Shaded}
\begin{Highlighting}[]
\FunctionTok{ssh}\NormalTok{ netid@cc{-}login.campuscluster.illinois.edu}
\CommentTok{\# password}
\end{Highlighting}
\end{Shaded}

Connecting in this manner is tedious since it involves repetitively
typing out login credentials. There are two tricks that void the
necessity to do so. Effectively, we have:

\begin{itemize}
\tightlist
\item
  Passwordless login

  \begin{itemize}
  \tightlist
  \item
    Public/Private SSH Keys
  \end{itemize}
\item
  Alias connection names

  \begin{itemize}
  \tightlist
  \item
    SSH Config
  \end{itemize}
\end{itemize}

Thus, instead of entering a password, the local computer can submit a
private key to be verified by a server. Not only is this more secure,
but it avoids the hassle of remembering the password and typing it out
while observers watch. Secondly, the connection alias will allow for
typing:

\begin{Shaded}
\begin{Highlighting}[]
\FunctionTok{ssh}\NormalTok{ icc}
\end{Highlighting}
\end{Shaded}

Not bad eh?

\hypertarget{generating-an-ssh-key}{%
\subsection{Generating an SSH Key}\label{generating-an-ssh-key}}

On your \textbf{local} computer, open up Terminal and type:

\begin{Shaded}
\begin{Highlighting}[]
\CommentTok{\#\# Run:}
\FunctionTok{ssh{-}keygen} \AttributeTok{{-}t}\NormalTok{ rsa }\AttributeTok{{-}C} \StringTok{"netid@illinois.edu"}
\CommentTok{\#\# Respond to:}
\CommentTok{\# Enter file in which to save the key (/home/demo/.ssh/id\_rsa): \# [Press enter]}
\CommentTok{\# Enter passphrase (empty for no passphrase): \# Write short password}
\end{Highlighting}
\end{Shaded}

\hypertarget{copy-ssh-key-to-server}{%
\subsection{Copy SSH Key to Server}\label{copy-ssh-key-to-server}}

Next, let's copy the generated key from your \textbf{local} computer
onto the cluster.

\begin{Shaded}
\begin{Highlighting}[]
\CommentTok{\#\# Run:}
\ExtensionTok{ssh{-}copy{-}id}\NormalTok{ netid@cc{-}login.campuscluster.illinois.edu}
\end{Highlighting}
\end{Shaded}

On macOS, prior to using \texttt{ssh-copy-id}, the command will need to
be installed. \href{https://brew.sh/}{\texttt{Homebrew}} provides a
formula that will setup the command. Install using:

\begin{Shaded}
\begin{Highlighting}[]
\CommentTok{\# Install homebrew}
\ExtensionTok{/bin/bash} \AttributeTok{{-}c} \StringTok{"}\VariableTok{$(}\ExtensionTok{curl} \AttributeTok{{-}fsSL}\NormalTok{ https://raw.githubusercontent.com/Homebrew/install/master/install.sh}\VariableTok{)}\StringTok{"}
\CommentTok{\# Install the required command binary}
\ExtensionTok{brew}\NormalTok{ install ssh{-}copy{-}id}
\end{Highlighting}
\end{Shaded}

\hypertarget{ssh-config-file}{%
\subsection{SSH Config File}\label{ssh-config-file}}

Inside of \texttt{\textasciitilde{}/.ssh/config}, add the following host
configuration. Make sure to \textbf{replace}
\texttt{\textless{}netid\textgreater{}} value with your personal netid.

\begin{Shaded}
\begin{Highlighting}[]
\ExtensionTok{Host}\NormalTok{ icc}
    \ExtensionTok{HostName}\NormalTok{ cc{-}login.campuscluster.illinois.edu}
    \ExtensionTok{User}\NormalTok{ netid}
\end{Highlighting}
\end{Shaded}

\textbf{Note:} This assumes a default location is used for the SSH key.
If there is a custom SSH key location add
\texttt{IdentityFile\ \textasciitilde{}/.ssh/sshkeyname.key} after the
\texttt{User} line.

\hypertarget{bash-aliases}{%
\section{Bash Aliases}\label{bash-aliases}}

Bash has the ability to create command aliases through \texttt{alias}.
The primary use is to take long commands and create short-cuts to avoid
typing them. Alternatively, this allows one to also rename commonly used
commands. For example, one could modify the \texttt{ls} command to
always list each file and show all hidden files with:

\begin{Shaded}
\begin{Highlighting}[]
\BuiltInTok{alias}\NormalTok{ ls=}\StringTok{\textquotesingle{}ls {-}la\textquotesingle{}}\KeywordTok{\textasciigrave{}} \BuiltInTok{.}
\end{Highlighting}
\end{Shaded}

We suggest creating a \texttt{\textasciitilde{}/.bash\_aliases} on the
cluster and filling it with:

\begin{Shaded}
\begin{Highlighting}[]
\ExtensionTok{{-}{-}8}\OperatorTok{\textless{}}\NormalTok{{-}{-} }\StringTok{"config/.bash\_aliases"}
\end{Highlighting}
\end{Shaded}

You may download this directly onto the cluster using:

\begin{Shaded}
\begin{Highlighting}[]
\FunctionTok{wget}\NormalTok{ https://raw.githubusercontent.com/coatless/hpc/master/docs/config/.bash\_aliases}
\end{Highlighting}
\end{Shaded}

To ensure bash aliases are available, we need to add the file to
\texttt{\textasciitilde{}/.bashrc}:

\begin{Shaded}
\begin{Highlighting}[]
\ExtensionTok{{-}{-}8}\OperatorTok{\textless{}}\NormalTok{{-}{-} }\StringTok{"config/.bashrc"}
\end{Highlighting}
\end{Shaded}

\textbf{Note:} the load modules component is shown

You may download this directly onto the cluster using:

\begin{Shaded}
\begin{Highlighting}[]
\FunctionTok{rm} \AttributeTok{{-}rf}\NormalTok{ \textasciitilde{}/.bashrc}
\FunctionTok{wget}\NormalTok{ https://raw.githubusercontent.com/coatless/hpc/master/docs/config/.bashrc}
\end{Highlighting}
\end{Shaded}

\hypertarget{optional-github-personal-access-token-pat}{%
\section{Optional: GitHub Personal Access Token
(PAT)}\label{optional-github-personal-access-token-pat}}

We briefly summarize the process for getting and registering a
\href{https://help.github.com/articles/creating-an-access-token-for-command-line-use/}{GitHub
Personal Access Token} in \emph{R}.

\begin{figure}

{\centering 

\href{https://www.youtube.com/watch?v=c14aqVC-Szo}{\includegraphics{https://img.youtube.com/vi/c14aqVC-Szo/0.jpg}}

}

\caption{PAT Token Walkthrough Video}

\end{figure}

The token may be created at: \url{https://github.com/settings/tokens}

From there, we can add it to the \emph{R} session with:

\begin{Shaded}
\begin{Highlighting}[]
\FunctionTok{touch}\NormalTok{ \textasciitilde{}/.Renviron}
\FunctionTok{cat} \OperatorTok{\textless{}\textless{} EOF} \OperatorTok{\textgreater{}\textgreater{}}\NormalTok{ \textasciitilde{}/.Renviron}
\StringTok{GITHUB\_TOKEN="your\_github\_token\_here"}
\StringTok{GITHUB\_PAT="your\_github\_token\_here"}
\OperatorTok{EOF}
\end{Highlighting}
\end{Shaded}

Alternatively, within \emph{R}, the token can be added by typing:

\begin{Shaded}
\begin{Highlighting}[]
\FunctionTok{file.edit}\NormalTok{(}\StringTok{"\textasciitilde{}/.Renviron"}\NormalTok{)}
\end{Highlighting}
\end{Shaded}

Then, writing in the configuration file:

\begin{Shaded}
\begin{Highlighting}[]
\VariableTok{GITHUB\_TOKEN}\OperatorTok{=}\StringTok{"your\_github\_token\_here"}
\VariableTok{GITHUB\_PAT}\OperatorTok{=}\StringTok{"your\_github\_token\_here"}
\end{Highlighting}
\end{Shaded}

\hypertarget{default-r-package-storage-location}{%
\section{\texorpdfstring{Default \emph{R} Package Storage
Location}{Default R Package Storage Location}}\label{default-r-package-storage-location}}

\emph{R}'s default library directory where packages are installed into
is found within the user's home directory at:

\begin{Shaded}
\begin{Highlighting}[]
\CommentTok{\# location for R 3.6.z}
\ExtensionTok{/home/}\VariableTok{$USER}\ExtensionTok{/R/x86\_64{-}pc{-}linux{-}gnu{-}library/3.6}

\CommentTok{\# location for R 4.0.z}
\ExtensionTok{/home/}\VariableTok{$USER}\ExtensionTok{/R/x86\_64{-}pc{-}linux{-}gnu{-}library/4.0}

\CommentTok{\# location for R 4.1.z}
\ExtensionTok{/home/}\VariableTok{$USER}\ExtensionTok{/R/x86\_64{-}pc{-}linux{-}gnu{-}library/4.1}
\end{Highlighting}
\end{Shaded}

Installing packages into the default location is problematic because any
files placed within a user's home directory count against the
directory's space quota (for limits, please see
(\textbf{cluster-storage?})). As \emph{R} packages can take a
considerable amount of space when installed, the best course of action
is to change the default library directory. Therefore, \emph{R} packages
should be either stored in a project directory or a purchased space
allocation on the cluster that an investor may purchase.

The path to an investor's space is given as:

\begin{Shaded}
\begin{Highlighting}[]
\ExtensionTok{/projects/}\OperatorTok{\textless{}}\NormalTok{investor}\OperatorTok{\textgreater{}}\NormalTok{/shared/}\VariableTok{$USER}
\end{Highlighting}
\end{Shaded}

Frequently, the cluster staff will create a symlink into the investor's
directory once authorization is given. In the case of
\textbf{Statistics}, the investor name is \texttt{stat}, so the
directory would be either:

\begin{Shaded}
\begin{Highlighting}[]
\ExtensionTok{/projects/stat/shared/}\VariableTok{$USER}
\CommentTok{\# or the symlink version:}
\ExtensionTok{\textasciitilde{}/project{-}stat/}
\end{Highlighting}
\end{Shaded}

In any case, we recommend creating and registering an \texttt{r-pkgs}
directory under the appropriate project space. The registration with
\emph{R} is done using the
\href{https://stat.ethz.ch/R-manual/R-patched/library/base/html/libPaths.html}{\texttt{R\_LIBS\_USER}
variable} in
\href{https://stat.ethz.ch/R-manual/R-patched/library/base/html/Startup.html}{\texttt{\textasciitilde{}/.Renvion}}.

\begin{Shaded}
\begin{Highlighting}[]
\CommentTok{\# Setup the .Renviron file in the home directory}
\FunctionTok{touch}\NormalTok{ \textasciitilde{}/.Renviron}

\CommentTok{\# Append a single variable into the Renvironment file}
\FunctionTok{cat} \OperatorTok{\textless{}\textless{} \textquotesingle{}EOF\textquotesingle{}} \OperatorTok{\textgreater{}\textgreater{}}\NormalTok{ \textasciitilde{}/.Renviron}
\StringTok{\# Location to R library}
\StringTok{R\_LIBS\_USER=\textasciitilde{}/project{-}stat/R/\%p{-}library/\%v}
\OperatorTok{EOF}

\CommentTok{\# Construct the path}
\ExtensionTok{Rscript} \AttributeTok{{-}e} \StringTok{\textquotesingle{}dir.create(Sys.getenv("R\_LIBS\_USER"), recursive = TRUE)\textquotesingle{}}
\end{Highlighting}
\end{Shaded}

Under this approach, we have move the location of the default package
directory to:

\begin{Shaded}
\begin{Highlighting}[]
\ExtensionTok{\textasciitilde{}/project{-}stat/R/\%p{-}library/\%v}
\CommentTok{\# the expanded version of \%p and \%v give:}
\ExtensionTok{\textasciitilde{}/project{-}stat/R/x86\_64{-}pc{-}linux{-}gnu{-}library/x.y}
\end{Highlighting}
\end{Shaded}

\textbf{Note:} After each minor \emph{R} version upgrade of R x.y, you
will need to recreate the package storage directory using:

\begin{Shaded}
\begin{Highlighting}[]
\NormalTok{Rscript }\SpecialCharTok{{-}}\NormalTok{e }\StringTok{\textquotesingle{}dir.create(Sys.getenv("R\_LIBS\_USER"), recursive = TRUE)\textquotesingle{}}
\end{Highlighting}
\end{Shaded}

One question that arises:

\begin{quote}
Why not set up a generic personal library directory called
\texttt{\textasciitilde{}/Rlibs}?
\end{quote}

We avoided a generic name for two reasons:

\begin{enumerate}
\def\labelenumi{\arabic{enumi}.}
\tightlist
\item
  New ``major'' releases of \emph{R} -- and sometime minor versions --
  are incompatible with the old packages.
\item
  Versioning by number allows for graceful downgrades if needed.
\end{enumerate}

In the case of the first bullet, its better to start over from a new
directory to ensure clean builds.

Though, you could opt not to and remember:

\begin{Shaded}
\begin{Highlighting}[]
\FunctionTok{update.packages}\NormalTok{(}\AttributeTok{ask =} \ConstantTok{FALSE}\NormalTok{, }\AttributeTok{checkBuilt =} \ConstantTok{TRUE}\NormalTok{)}
\end{Highlighting}
\end{Shaded}

\hypertarget{install-r-packages-into-library}{%
\section{\texorpdfstring{Install \emph{R} packages into
library}{Install R packages into library}}\label{install-r-packages-into-library}}

Prior to installing an \emph{R} package, make sure to load the
appropriate \emph{R} version with:

\begin{Shaded}
\begin{Highlighting}[]
\NormalTok{module load R}\SpecialCharTok{/}\NormalTok{x.y.z}
\end{Highlighting}
\end{Shaded}

where \texttt{x.y.z} is a supported version number,
e.g.~\texttt{module\ load\ R/4.1.1\_sandybridge} will make available
\emph{R} 4.1.1 that works on any cluster node.

Once \emph{R} is loaded, packages can be installed by entering into
\emph{R} or directly from bash. The prior approach will be preferred as
it mimics local \emph{R} installation procedures while the latter
approach is useful for one-off packages installations.

Enter into an interactive \emph{R} session from bash by typing:

\begin{Shaded}
\begin{Highlighting}[]
\ExtensionTok{R}
\end{Highlighting}
\end{Shaded}

Then, inside of \emph{R}, the package installation may be done using:

\begin{Shaded}
\begin{Highlighting}[]
\CommentTok{\# Install a package}
\FunctionTok{install.packages}\NormalTok{(}\StringTok{\textquotesingle{}remotes\textquotesingle{}}\NormalTok{, }\AttributeTok{repos =} \StringTok{\textquotesingle{}https://cloud.r{-}project.org\textquotesingle{}}\NormalTok{)}

\CommentTok{\# Exit out of R and return to bash.}
\FunctionTok{q}\NormalTok{(}\AttributeTok{save =} \StringTok{"no"}\NormalTok{)}
\end{Highlighting}
\end{Shaded}

Unlike the native \emph{R} installation route, installing packages under
bash uses the \texttt{Rscript} command and requires writing the install
command as a string:

\begin{Shaded}
\begin{Highlighting}[]
\ExtensionTok{Rscript} \AttributeTok{{-}e} \StringTok{"install.packages(\textquotesingle{}remotes\textquotesingle{}, repos = \textquotesingle{}https://cloud.r{-}project.org\textquotesingle{})"}
\end{Highlighting}
\end{Shaded}

Be careful when using quotations to specify packages. For each of these
commands, we begin and end with \texttt{"} and, thus, inside the command
we use \texttt{\textquotesingle{}} to denote strings. With this
approach, escaping character strings is avoided.

\hypertarget{installing-packages-into-development-libraries}{%
\subsection{Installing Packages into Development
Libraries}\label{installing-packages-into-development-libraries}}

If you need to use a different library path than what was setup as the
default, e.g.~\texttt{\textasciitilde{}/project-stat/r-libs}, first
create the directory and, then, specify a path to it with
\texttt{lib\ =\ \textquotesingle{}\textquotesingle{}} in
`install.packages().

\begin{Shaded}
\begin{Highlighting}[]
\FunctionTok{mkdir} \AttributeTok{{-}p}\NormalTok{ \textasciitilde{}/project{-}stat/devel{-}pkg}
\ExtensionTok{Rscript} \AttributeTok{{-}e} \StringTok{"install.packages(\textquotesingle{}remotes\textquotesingle{}, lib = \textquotesingle{}\textasciitilde{}/project{-}stat/devel{-}pkg\textquotesingle{},}
\StringTok{                             repos = \textquotesingle{}https://cloud.r{-}project.org/\textquotesingle{})"}
\end{Highlighting}
\end{Shaded}

\hypertarget{installing-packages-from-github}{%
\subsection{Installing Packages from
GitHub}\label{installing-packages-from-github}}

For packages stored on GitHub, there are two variants for installation
depending on the state of the repository. If the repository is
\textbf{public}, then the standard \texttt{install\_github("user/repo")}
may be used. On the other hand, if the repository is \textbf{private},
the package installation call must be accompanied by a
\href{https://help.github.com/articles/creating-an-access-token-for-command-line-use/}{\textbf{GitHub
Personal Access Token}} in the
\texttt{auth\_token=\textquotesingle{}\textquotesingle{}} parameter of
\texttt{install\_github()}. In the prior step, if the
\texttt{\textasciitilde{}/.Renviron} contains \texttt{GITHUB\_PAT}
variable, there is no need to specify in the \texttt{install\_github()}
call as it will automatically be picked up.

\begin{verbatim}
# Install package from GitHub
Rscript -e "remotes::install_github('coatless/visualize')"

# Install from a private repository on GitHub
Rscript -e "remotes::install_github('stat385/netid',
                                     auth_token = 'abc')"
\end{verbatim}

\hypertarget{parallelized-package-installation}{%
\subsection{Parallelized package
installation}\label{parallelized-package-installation}}

By default, all users are placed onto the login nodes. Login nodes are
configured for staging and submitting jobs \emph{not} for installing
software. The best practice and absolute fastest way to install software
is to use an \textbf{interactive job}. Interactive jobs place the user
directly on a compute node with the requested resources, e.g.~10 CPUs or
5GB of memory per CPU.

Before installing multiple \emph{R} packages, we recommend creating an
interactive job with:

\begin{Shaded}
\begin{Highlighting}[]
\ExtensionTok{srun} \AttributeTok{{-}{-}cpus{-}per{-}task}\OperatorTok{=}\NormalTok{10 }\AttributeTok{{-}{-}pty}\NormalTok{ bash}
\end{Highlighting}
\end{Shaded}

Once on the interactive node, load the appropriate version of \emph{R}:

\begin{Shaded}
\begin{Highlighting}[]
\NormalTok{module load R}\SpecialCharTok{/}\NormalTok{x.y.z }\CommentTok{\# where x.y.z is the version number}
\end{Highlighting}
\end{Shaded}

From here, make sure every package installation call uses the
\texttt{Ncpus\ =} parameter set equal to the number of cores requested
for the interactive job.

\begin{Shaded}
\begin{Highlighting}[]
\NormalTok{Rscript }\SpecialCharTok{{-}}\NormalTok{e }\StringTok{"install.packages(\textquotesingle{}remotes\textquotesingle{}, repos = \textquotesingle{}https://cloud.r{-}project.org\textquotesingle{}, Ncpus = 10L)"}
\end{Highlighting}
\end{Shaded}




\end{document}
